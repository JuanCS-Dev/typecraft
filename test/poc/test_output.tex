% Options for packages loaded elsewhere
\PassOptionsToPackage{unicode}{hyperref}
\PassOptionsToPackage{hyphens}{url}
%
\documentclass[
]{book}
\usepackage{amsmath,amssymb}
\usepackage{iftex}
\ifPDFTeX
  \usepackage[T1]{fontenc}
  \usepackage[utf8]{inputenc}
  \usepackage{textcomp} % provide euro and other symbols
\else % if luatex or xetex
  \usepackage{unicode-math} % this also loads fontspec
  \defaultfontfeatures{Scale=MatchLowercase}
  \defaultfontfeatures[\rmfamily]{Ligatures=TeX,Scale=1}
\fi
\usepackage{lmodern}
\ifPDFTeX\else
  % xetex/luatex font selection
\fi
% Use upquote if available, for straight quotes in verbatim environments
\IfFileExists{upquote.sty}{\usepackage{upquote}}{}
\IfFileExists{microtype.sty}{% use microtype if available
  \usepackage[]{microtype}
  \UseMicrotypeSet[protrusion]{basicmath} % disable protrusion for tt fonts
}{}
\makeatletter
\@ifundefined{KOMAClassName}{% if non-KOMA class
  \IfFileExists{parskip.sty}{%
    \usepackage{parskip}
  }{% else
    \setlength{\parindent}{0pt}
    \setlength{\parskip}{6pt plus 2pt minus 1pt}}
}{% if KOMA class
  \KOMAoptions{parskip=half}}
\makeatother
\usepackage{xcolor}
\usepackage[margin=1in]{geometry}
\setlength{\emergencystretch}{3em} % prevent overfull lines
\providecommand{\tightlist}{%
  \setlength{\itemsep}{0pt}\setlength{\parskip}{0pt}}
\setcounter{secnumdepth}{-\maxdimen} % remove section numbering
\ifLuaTeX
  \usepackage{selnolig}  % disable illegal ligatures
\fi
\IfFileExists{bookmark.sty}{\usepackage{bookmark}}{\usepackage{hyperref}}
\IfFileExists{xurl.sty}{\usepackage{xurl}}{} % add URL line breaks if available
\urlstyle{same}
\hypersetup{
  pdftitle={Livro de Teste - Typecraft},
  pdfauthor={Juan (Arquiteto-Chefe)},
  hidelinks,
  pdfcreator={LaTeX via pandoc}}

\title{Livro de Teste - Typecraft}
\author{Juan (Arquiteto-Chefe)}
\date{31 de Outubro de 2025}

\begin{document}
\frontmatter
\maketitle

\mainmatter
\hypertarget{capuxedtulo-1-o-inuxedcio}{%
\chapter{Capítulo 1: O Início}\label{capuxedtulo-1-o-inuxedcio}}

Este é um \textbf{teste} do sistema Typecraft. O objetivo é validar o
pipeline completo:

\begin{enumerate}
\def\labelenumi{\arabic{enumi}.}
\tightlist
\item
  Conversão de Markdown → LaTeX
\item
  Renderização LaTeX → PDF
\item
  Aplicação de princípios tipográficos
\end{enumerate}

\hypertarget{seuxe7uxe3o-1.1-tipografia-cluxe1ssica}{%
\section{Seção 1.1: Tipografia
Clássica}\label{seuxe7uxe3o-1.1-tipografia-cluxe1ssica}}

A tipografia existe para \emph{honrar o conteúdo}, como ensinou Robert
Bringhurst.

\hypertarget{subsection-teste-de-hierarquia}{%
\subsection{Subsection: Teste de
Hierarquia}\label{subsection-teste-de-hierarquia}}

Este é um parágrafo de teste com texto suficiente para validar:

\begin{itemize}
\tightlist
\item
  Quebra de linha
\item
  Espaçamento entre palavras
\item
  Entrelinha (leading)
\item
  Margens harmoniosas
\end{itemize}

\hypertarget{capuxedtulo-2-teste-de-estrutura}{%
\chapter{Capítulo 2: Teste de
Estrutura}\label{capuxedtulo-2-teste-de-estrutura}}

Lorem ipsum dolor sit amet, consectetur adipiscing elit. Sed do eiusmod
tempor incididunt ut labore et dolore magna aliqua.

\hypertarget{seuxe7uxe3o-2.1-mais-conteuxfado}{%
\section{Seção 2.1: Mais
Conteúdo}\label{seuxe7uxe3o-2.1-mais-conteuxfado}}

Ut enim ad minim veniam, quis nostrud exercitation ullamco laboris nisi
ut aliquip ex ea commodo consequat.

\begin{center}\rule{0.5\linewidth}{0.5pt}\end{center}

\textbf{Fim do documento de teste.}

\backmatter
\end{document}
